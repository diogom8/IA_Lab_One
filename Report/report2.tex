\documentclass[11pt,journal,compsoc]{IEEEtran}

\usepackage[utf8]{inputenc}
\usepackage{todonotes}

\usepackage{tikz}
\usetikzlibrary{shapes,arrows}

% *** CITATION PACKAGES ***
%
\ifCLASSOPTIONcompsoc
  % IEEE Computer Society needs nocompress option
  % requires cite.sty v4.0 or later (November 2003)
  % \usepackage[nocompress]{cite}
\else
  % normal IEEE
  % \usepackage{cite}
\fi


% *** GRAPHICS RELATED PACKAGES ***
%
\ifCLASSINFOpdf
  % \usepackage[pdftex]{graphicx}
  % declare the path(s) where your graphic files are
  % \graphicspath{{../pdf/}{../jpeg/}}
  % and their extensions so you won't have to specify these with
  % every instance of \includegraphics
  % \DeclareGraphicsExtensions{.pdf,.jpeg,.png}
\else
  % or other class option (dvipsone, dvipdf, if not using dvips). graphicx
  % will default to the driver specified in the system graphics.cfg if no
  % driver is specified.
  % \usepackage[dvips]{graphicx}
  % declare the path(s) where your graphic files are
  % \graphicspath{{../eps/}}
  % and their extensions so you won't have to specify these with
  % every instance of \includegraphics
  % \DeclareGraphicsExtensions{.eps}
\fi

% *** TITLE/SUBJECT/AUTHOR/KEYWORDS INFO BELOW!!           ***
\newcommand\MYhyperrefoptions{bookmarks=true,bookmarksnumbered=true,
pdfpagemode={UseOutlines},plainpages=false,pdfpagelabels=true,
colorlinks=true,linkcolor={black},citecolor={black},urlcolor={black},
pdftitle={Bare Demo of IEEEtran.cls for Computer Society Journals},%<!CHANGE!
pdfsubject={Typesetting},%<!CHANGE!
pdfauthor={Michael D. Shell},%<!CHANGE!
pdfkeywords={Computer Society, IEEEtran, journal, LaTeX, paper,
             template}}%<^!CHANGE!

% correct bad hyphenation here
\hyphenation{op-tical net-works semi-conduc-tor}


\begin{document}
%
% paper title
\title{Artificial Intelligence and Decision Systems: Assignment \#1}


\author{D. Monteiro$^\dagger$\thanks{$^\dagger$ MSc. student of Aerospace Engineering, Instituto Superior Técnico, Lisbon, Portugal. Student Nr: $70125$}, R.F. Santos$^\ddagger$\thanks{$^\ddagger$ MSc. student of Aerospace Engineering, Instituto Superior Técnico, Lisbon, Portugal. Student Nr: $69278$}\\[.2 cm]
\textit{University of Lisbon, Lisbon, Portugal}}% <-this % stops a space
     
        
%\IEEEcompsocitemizethanks{\IEEEcompsocthanksitem D. Monteiro, Aerospace Engineering Student in Instituto Superior Técnico, Lisbon, Portugal. Number 70125.\protect\\
% note need leading \protect in front of \\ to get a newline within \thanks as
% \\ is fragile and will error, could use \hfil\break instead.
%E-mail: diogo\_monteiro\_1$@$hotmail.com

%\IEEEcompsocthanksitem R. Santos, Aerospace Engineering Student in Instituto Superior Técnico, Lisbon, Portugal. Number 69278.\protect\\
% note need leading \protect in front of \\ to get a newline within \thanks as
% \\ is fragile and will error, could use \hfil\break instead.
%E-mail: rafael.j.f.santos$@$tecnico.ulisboa.pt}}

% note the % following the last \IEEEmembership and also \thanks - 
% these prevent an unwanted space from occurring between the last author name
% and the end of the author line. i.e., if you had this:
% 
% \author{....lastname \thanks{...} \thanks{...} }
%                     ^------------^------------^----Do not want these spaces!
%
% a space would be appended to the last name and could cause every name on that
% line to be shifted left slightly. This is one of those "LaTeX things". For
% instance, "\textbf{A} \textbf{B}" will typeset as "A B" not "AB". To get
% "AB" then you have to do: "\textbf{A}\textbf{B}"
% \thanks is no different in this regard, so shield the last } of each \thanks
% that ends a line with a % and do not let a space in before the next \thanks.
% Spaces after \IEEEmembership other than the last one are OK (and needed) as
% you are supposed to have spaces between the names. For what it is worth,
% this is a minor point as most people would not even notice if the said evil
% space somehow managed to creep in.



% The paper headers
%\markboth{Journal of \LaTeX\ Class Files,~Vol.~11, No.~4, December~2012}%
%{Shell \MakeLowercase{\textit{et al.}}: Bare Advanced Demo of IEEEtran.cls for Journals}

%\markboth{wadaw}%
%Shell \MakeLowercase{\textit{et al.}}: Artificial Intelligence and Decision Systems}


% The only time the second header will appear is for the odd numbered pages
% after the title page when using the twoside option.
% 
% *** Note that you probably will NOT want to include the author's ***
% *** name in the headers of peer review papers.                   ***
% You can use \ifCLASSOPTIONpeerreview for conditional compilation here if
% you desire.



% The publisher's ID mark at the bottom of the page is less important with
% Computer Society journal papers as those publications place the marks
% outside of the main text columns and, therefore, unlike regular IEEE
% journals, the available text space is not reduced by their presence.
% If you want to put a publisher's ID mark on the page you can do it like
% this:
%\IEEEpubid{0000--0000/00\$00.00~\copyright~2012 IEEE}
% or like this to get the Computer Society new two part style.
%\IEEEpubid{\makebox[\columnwidth]{\hfill 0000--0000/00/\$00.00~\copyright~2012 IEEE}%
%\hspace{\columnsep}\makebox[\columnwidth]{Published by the IEEE Computer Society\hfill}}
% Remember, if you use this you must call \IEEEpubidadjcol in the second
% column for its text to clear the IEEEpubid mark (Computer Society journal
% papers don't need this extra clearance.)



% use for special paper notices
%\IEEEspecialpapernotice{(Invited Paper)}



% for Computer Society papers, we must declare the abstract and index terms
% PRIOR to the title within the \IEEEtitleabstractindextext IEEEtran
% command as these need to go into the title area created by \maketitle.
% As a general rule, do not put math, special symbols or citations
% in the abstract or keywords.
\IEEEtitleabstractindextext{%
\begin{abstract}
The abstract goes here.
\end{abstract}

% Note that keywords are not normally used for peerreview papers.
\begin{IEEEkeywords}
Computer Society, IEEEtran, journal, \LaTeX, paper, template.
\end{IEEEkeywords}}


% make the title area
\maketitle


% To allow for easy dual compilation without having to reenter the
% abstract/keywords data, the \IEEEtitleabstractindextext text will
% not be used in maketitle, but will appear (i.e., to be "transported")
% here as \IEEEdisplaynontitleabstractindextext when compsoc mode
% is not selected <OR> if conference mode is selected - because compsoc
% conference papers position the abstract like regular (non-compsoc)
% papers do!
\IEEEdisplaynontitleabstractindextext
% \IEEEdisplaynontitleabstractindextext has no effect when using
% compsoc under a non-conference mode.


% For peer review papers, you can put extra information on the cover
% page as needed:
% \ifCLASSOPTIONpeerreview
% \begin{center} \bfseries EDICS Category: 3-BBND \end{center}
% \fi
%
% For peerreview papers, this IEEEtran command inserts a page break and
% creates the second title. It will be ignored for other modes.
\IEEEpeerreviewmaketitle



\section{Introduction}
% Computer Society journal papers do something a tad strange with the very
% first section heading (almost always called "Introduction"). They place it
% ABOVE the main text! IEEEtran.cls currently does not do this for you.
% However, You can achieve this effect by making LaTeX jump through some
% hoops via something like:
%
%\ifCLASSOPTIONcompsoc
%  \noindent\raisebox{2\baselineskip}[0pt][0pt]%
%  {\parbox{\columnwidth}{\section{Introduction}\label{sec:introduction}%
%  \global\everypar=\everypar}}%
%  \vspace{-1\baselineskip}\vspace{-\parskip}\par
%\else
%  \section{Introduction}\label{sec:introduction}\par
%\fi
%
% Admittedly, this is a hack and may well be fragile, but seems to do the
% trick for me. Note the need to keep any \label that may be used right
% after \section in the above as the hack puts \section within a raised box.



% The very first letter is a 2 line initial drop letter followed
% by the rest of the first word in caps (small caps for compsoc).
% 
% form to use if the first word consists of a single letter:
% \IEEEPARstart{A}{demo} file is ....
% 
% form to use if you need the single drop letter followed by
% normal text (unknown if ever used by IEEE):
% \IEEEPARstart{A}{}demo file is ....
% 
% Some journals put the first two words in caps:
% \IEEEPARstart{T}{his demo} file is ....
% 
% Here we have the typical use of a "T" for an initial drop letter
% and "HIS" in caps to complete the first word.
\IEEEPARstart{T}{he} aim of the present short report is to describe the results obtained for the Assignment \#1 which aimed the exploration of both uninformed and informed search methods for a maze solving problem. A discussion considering the choices made along the assignment is performed along Sections 2 and 3. In Section 4 the report focuses on the implementation procedure as well in problem representation applied. Section 5 is devoted to the results comparison and final conclusions. The created code can be found attached. \textsc{Python} v2.7 released for \textsc{Unix} operating systems.

\hfill 
 
\hfill October 17, 2014
\section{Uninformed Search}

% Define block styles
\tikzstyle{initBlock} = [draw, draw, fill=blue!20, 
    text width=4.5em, text badly centered, node distance=3cm, inner sep=0pt, minimum width=6em, minimum height=5em]
\tikzstyle{block} = [rectangle, draw, fill=green!20, 
    text width=5em, text centered, rounded corners, minimum height=4em]
\tikzstyle{largerblock} = [rectangle, draw, fill=green!20, 
    text width=6em, text centered, rounded corners, minimum height=4em]    
\tikzstyle{line} = [draw, -latex']
\tikzstyle{cloud} = [draw, ellipse,fill=blue!20, node distance=3cm,
    minimum height=2em]
    
\begin{tikzpicture}[node distance = 2cm, auto]
    % Place nodes
    \node [initBlock] (init) {Move 'not' inwards function};
    \node [block, below of=init] (sentatomic) {Is sentence atomic?};
    \node [cloud, right of=sentatomic, node distance=6cm] (return) {return};
    \node [block, below of=sentatomic] (opnot) {Operator = 'not'?};
    \node [block, below of=opnot] (lowSent) {Is lower sentence atomic?};
    %\node [block, below of=lowSent] (lowOpNot) {Lower Operator = 'not'?};
	%\node [block, below of=lowOpNot] (lowOpAnd) {Lower Operator = 'and'?};
	%\node [block, below of=lowOpAnd] (lowOpOr) {Lower Operator = 'or'};
    %\node [block, right of=lowOpNot, node distance=3cm] (notnotA) {$\neg\neg A = A$};
	%\node [block, right of=lowOpAnd, node distance=3cm] (morgan1) {$\neg (A \wedge B) = (\neg A \vee \neg B)$};
	%\node [block, right of=lowOpOr, node distance=3cm] (morgan2) {$\neg (A \vee B) = (\neg A \wedge \neg B)$};
	\node [block, right of=opnot, node distance=3cm] (sents) {Check if left and right side sentences are atomic.};
	\node [cloud, right of=lowSent, node distance=6cm] (recursive) {recursive};
	\node [largerblock, below of=lowSent, node distance=2.5cm] (lowOpAndOperations) {Check lower operator and do necessary operations};

    % Draw edges
    \path [line] (init) -- (sentatomic);
    \path [line] (sentatomic) -- node {no} (opnot);
    \path [line] (sentatomic) -- node {yes} (return);
    \path [line] (opnot) -- node {yes} (lowSent);
    \path [line] (opnot) -- node {no} (sents);
    	\path [line] (sents) -- node {yes} (return);
    	\path [line] (sents) -- node {no} (recursive);
    \path [line] (lowSent) -- node {yes} (lowOpAndOperations);
    \path [line] (lowSent) -- node {no} (recursive);
    \path [line] (lowOpAndOperations) -|(recursive);
	%\path [line] (lowOpNot) --  node {no} (lowOpAnd);
	%\path [line] (lowOpAnd) --  node {no} (lowOpOr);
    %\path [line] (lowOpNot) -- node {yes} (notnotA);
    %\path [line] (lowOpAnd) -- node {yes} (morgan1);
    %\path [line] (lowOpOr) -- (morgan2);
    %\path [line] (notnotA) -| (recursive);
	%\path [line] (morgan1) -| (recursive);
	%\path [line] (morgan2) -| (recursive);
	\path [line] (recursive) -- (return);
\end{tikzpicture}




% references section

% can use a bibliography generated by BibTeX as a .bbl file
% BibTeX documentation can be easily obtained at:
% http://www.ctan.org/tex-archive/biblio/bibtex/contrib/doc/
% The IEEEtran BibTeX style support page is at:
% http://www.michaelshell.org/tex/ieeetran/bibtex/
%\bibliographystyle{IEEEtran}
% argument is your BibTeX string definitions and bibliography database(s)
%\bibliography{IEEEabrv,../bib/paper}
%
% <OR> manually copy in the resultant .bbl file
% set second argument of \begin to the number of references
% (used to reserve space for the reference number labels box)
\begin{thebibliography}{1}

\bibitem{book}
S.~Russell and P.~Norvig, \emph{Atificial Intelligence - A Modern Approach}, 3rd~ed.\hskip 1em plus
  0.5em minus 0.4em\relax Pearson, 2010.

\end{thebibliography}

% biography section
% 
% If you have an EPS/PDF photo (graphicx package needed) extra braces are
% needed around the contents of the optional argument to biography to prevent
% the LaTeX parser from getting confused when it sees the complicated
% \includegraphics command within an optional argument. (You could create
% your own custom macro containing the \includegraphics command to make things
% simpler here.)
%\begin{IEEEbiography}[{\includegraphics[width=1in,height=1.25in,clip,keepaspectratio]{mshell}}]{Michael Shell}
% or if you just want to reserve a space for a photo:


\end{document}


