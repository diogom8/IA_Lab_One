\documentclass[11pt,journal,compsoc]{IEEEtran}

\usepackage[utf8]{inputenc}
\usepackage{todonotes}

% *** CITATION PACKAGES ***
%
\ifCLASSOPTIONcompsoc
  % IEEE Computer Society needs nocompress option
  % requires cite.sty v4.0 or later (November 2003)
  % \usepackage[nocompress]{cite}
\else
  % normal IEEE
  % \usepackage{cite}
\fi


% *** GRAPHICS RELATED PACKAGES ***
%
\ifCLASSINFOpdf
  % \usepackage[pdftex]{graphicx}
  % declare the path(s) where your graphic files are
  % \graphicspath{{../pdf/}{../jpeg/}}
  % and their extensions so you won't have to specify these with
  % every instance of \includegraphics
  % \DeclareGraphicsExtensions{.pdf,.jpeg,.png}
\else
  % or other class option (dvipsone, dvipdf, if not using dvips). graphicx
  % will default to the driver specified in the system graphics.cfg if no
  % driver is specified.
  % \usepackage[dvips]{graphicx}
  % declare the path(s) where your graphic files are
  % \graphicspath{{../eps/}}
  % and their extensions so you won't have to specify these with
  % every instance of \includegraphics
  % \DeclareGraphicsExtensions{.eps}
\fi

% *** TITLE/SUBJECT/AUTHOR/KEYWORDS INFO BELOW!!           ***
\newcommand\MYhyperrefoptions{bookmarks=true,bookmarksnumbered=true,
pdfpagemode={UseOutlines},plainpages=false,pdfpagelabels=true,
colorlinks=true,linkcolor={black},citecolor={black},urlcolor={black},
pdftitle={Bare Demo of IEEEtran.cls for Computer Society Journals},%<!CHANGE!
pdfsubject={Typesetting},%<!CHANGE!
pdfauthor={Michael D. Shell},%<!CHANGE!
pdfkeywords={Computer Society, IEEEtran, journal, LaTeX, paper,
             template}}%<^!CHANGE!

% correct bad hyphenation here
\hyphenation{op-tical net-works semi-conduc-tor}


\begin{document}
%
% paper title
\title{Artificial Intelligence and Decision Systems: Assignment \#1}


\author{D. Monteiro$^\dagger$\thanks{$^\dagger$ MSc. student of Aerospace Engineering, Instituto Superior Técnico, Lisbon, Portugal. Student Nr: $70125$}, R.F. Santos$^\ddagger$\thanks{$^\ddagger$ MSc. student of Aerospace Engineering, Instituto Superior Técnico, Lisbon, Portugal. Student Nr: $69278$}\\[.2 cm]
\textit{University of Lisbon, Lisbon, Portugal}}% <-this % stops a space
     
        
%\IEEEcompsocitemizethanks{\IEEEcompsocthanksitem D. Monteiro, Aerospace Engineering Student in Instituto Superior Técnico, Lisbon, Portugal. Number 70125.\protect\\
% note need leading \protect in front of \\ to get a newline within \thanks as
% \\ is fragile and will error, could use \hfil\break instead.
%E-mail: diogo\_monteiro\_1$@$hotmail.com

%\IEEEcompsocthanksitem R. Santos, Aerospace Engineering Student in Instituto Superior Técnico, Lisbon, Portugal. Number 69278.\protect\\
% note need leading \protect in front of \\ to get a newline within \thanks as
% \\ is fragile and will error, could use \hfil\break instead.
%E-mail: rafael.j.f.santos$@$tecnico.ulisboa.pt}}

% note the % following the last \IEEEmembership and also \thanks - 
% these prevent an unwanted space from occurring between the last author name
% and the end of the author line. i.e., if you had this:
% 
% \author{....lastname \thanks{...} \thanks{...} }
%                     ^------------^------------^----Do not want these spaces!
%
% a space would be appended to the last name and could cause every name on that
% line to be shifted left slightly. This is one of those "LaTeX things". For
% instance, "\textbf{A} \textbf{B}" will typeset as "A B" not "AB". To get
% "AB" then you have to do: "\textbf{A}\textbf{B}"
% \thanks is no different in this regard, so shield the last } of each \thanks
% that ends a line with a % and do not let a space in before the next \thanks.
% Spaces after \IEEEmembership other than the last one are OK (and needed) as
% you are supposed to have spaces between the names. For what it is worth,
% this is a minor point as most people would not even notice if the said evil
% space somehow managed to creep in.



% The paper headers
%\markboth{Journal of \LaTeX\ Class Files,~Vol.~11, No.~4, December~2012}%
%{Shell \MakeLowercase{\textit{et al.}}: Bare Advanced Demo of IEEEtran.cls for Journals}

%\markboth{wadaw}%
%Shell \MakeLowercase{\textit{et al.}}: Artificial Intelligence and Decision Systems}


% The only time the second header will appear is for the odd numbered pages
% after the title page when using the twoside option.
% 
% *** Note that you probably will NOT want to include the author's ***
% *** name in the headers of peer review papers.                   ***
% You can use \ifCLASSOPTIONpeerreview for conditional compilation here if
% you desire.



% The publisher's ID mark at the bottom of the page is less important with
% Computer Society journal papers as those publications place the marks
% outside of the main text columns and, therefore, unlike regular IEEE
% journals, the available text space is not reduced by their presence.
% If you want to put a publisher's ID mark on the page you can do it like
% this:
%\IEEEpubid{0000--0000/00\$00.00~\copyright~2012 IEEE}
% or like this to get the Computer Society new two part style.
%\IEEEpubid{\makebox[\columnwidth]{\hfill 0000--0000/00/\$00.00~\copyright~2012 IEEE}%
%\hspace{\columnsep}\makebox[\columnwidth]{Published by the IEEE Computer Society\hfill}}
% Remember, if you use this you must call \IEEEpubidadjcol in the second
% column for its text to clear the IEEEpubid mark (Computer Society journal
% papers don't need this extra clearance.)



% use for special paper notices
%\IEEEspecialpapernotice{(Invited Paper)}



% for Computer Society papers, we must declare the abstract and index terms
% PRIOR to the title within the \IEEEtitleabstractindextext IEEEtran
% command as these need to go into the title area created by \maketitle.
% As a general rule, do not put math, special symbols or citations
% in the abstract or keywords.
\IEEEtitleabstractindextext{%
\begin{abstract}
The abstract goes here.
\end{abstract}

% Note that keywords are not normally used for peerreview papers.
\begin{IEEEkeywords}
Computer Society, IEEEtran, journal, \LaTeX, paper, template.
\end{IEEEkeywords}}


% make the title area
\maketitle


% To allow for easy dual compilation without having to reenter the
% abstract/keywords data, the \IEEEtitleabstractindextext text will
% not be used in maketitle, but will appear (i.e., to be "transported")
% here as \IEEEdisplaynontitleabstractindextext when compsoc mode
% is not selected <OR> if conference mode is selected - because compsoc
% conference papers position the abstract like regular (non-compsoc)
% papers do!
\IEEEdisplaynontitleabstractindextext
% \IEEEdisplaynontitleabstractindextext has no effect when using
% compsoc under a non-conference mode.


% For peer review papers, you can put extra information on the cover
% page as needed:
% \ifCLASSOPTIONpeerreview
% \begin{center} \bfseries EDICS Category: 3-BBND \end{center}
% \fi
%
% For peerreview papers, this IEEEtran command inserts a page break and
% creates the second title. It will be ignored for other modes.
\IEEEpeerreviewmaketitle



\section{Introduction}
% Computer Society journal papers do something a tad strange with the very
% first section heading (almost always called "Introduction"). They place it
% ABOVE the main text! IEEEtran.cls currently does not do this for you.
% However, You can achieve this effect by making LaTeX jump through some
% hoops via something like:
%
%\ifCLASSOPTIONcompsoc
%  \noindent\raisebox{2\baselineskip}[0pt][0pt]%
%  {\parbox{\columnwidth}{\section{Introduction}\label{sec:introduction}%
%  \global\everypar=\everypar}}%
%  \vspace{-1\baselineskip}\vspace{-\parskip}\par
%\else
%  \section{Introduction}\label{sec:introduction}\par
%\fi
%
% Admittedly, this is a hack and may well be fragile, but seems to do the
% trick for me. Note the need to keep any \label that may be used right
% after \section in the above as the hack puts \section within a raised box.



% The very first letter is a 2 line initial drop letter followed
% by the rest of the first word in caps (small caps for compsoc).
% 
% form to use if the first word consists of a single letter:
% \IEEEPARstart{A}{demo} file is ....
% 
% form to use if you need the single drop letter followed by
% normal text (unknown if ever used by IEEE):
% \IEEEPARstart{A}{}demo file is ....
% 
% Some journals put the first two words in caps:
% \IEEEPARstart{T}{his demo} file is ....
% 
% Here we have the typical use of a "T" for an initial drop letter
% and "HIS" in caps to complete the first word.
\IEEEPARstart{T}{he} aim of the present short report is to describe the results obtained for the Assignment \#1 which aimed the exploration of both uninformed and informed search methods for a maze solving problem. A discussion considering the choices made along the assignment is performed along Sections 2 and 3. In Section 4 the report focuses on the implementation procedure as well in problem representation applied. Section 5 is devoted to the results comparison and final conclusions. The created code can be found attached. \textsc{Python} v2.7 released for \textsc{Unix} operating systems.

\hfill 
 
\hfill October 17, 2014
\section{Uninformed Search}
The first implementation of the maze solver problem was based on the general uninformed search strategies, which are confined to information provided by the problem description. Although there are several different methods of this type, they all search blindly for the goal node. The distinction point within strategies lies on the order the successor nodes are generated.
The first methods to be discarded from this analysis was the depth-search method.	Despite being attractive due its lower memory use, no optimality is ensured. The uniform-cost method was also discarded since in our problem the path cost for each step is constant. The third non-considered method was the Bidirectional search which clearly presents advantages in terms of space and time complexity and optimality but suggests a rather challenging implementation. Moreover, considering that maps can have closed doors at any point it is easy to think of many examples to which this strategy would not bring any point on favour.\\
Considering all this aspects and keeping in mind that an optimal algorithm was required, the methods implemented were the \textbf{breadth-first} and \textbf{iterative deepening} search methods. Despite the fact that the time complexity expected for both methods is equal to $O(b^d)$, a decrease of space usage can theoretically be obtained in favour of the latter method.

\section{Informed Search}
The informed search strategies supposes the use of specific knowledge about the problem and not only its definition. With this more efficiency in terms of time and space complexity can be expected.\\
In this section the chosen method was the \textbf{A* algorithm} with a well suited heuristic function $h(n)$. This function combined with the path cost function $g(n)$ are the main responsible for the success of the search implementation. In order to ensure optimality the heuristic needs to be consistent implying admissibility. Consistency through a triangular inequality analysis. Those two properties of the heuristic function ensure that the search is optimal as intended.\\
The choice for an heuristic was not straight forward due to problem characteristics under analysis. The use of a distance-based heuristic could be fairly easy to implement, nevertheless that does not provide a completely trustable information about the way the agent should move. Suppose a scenario in which an agent is at two nodes distance from the goal node but between them there is a closed door which correspondent switch is on the opposite side of the maze. However, one shall also reason that building a rather complex heuristic\footnote{ For instance, suppose using an intermediate breadth-first search algorithm capable of providing the approximate length of the path for a simplified version of the maze} which provides a better indication can also mean a significant increase of the computation time. Having into consideration the referred aspects, the final choice was for the use of the well known \textit{Manhattan Distance} both admissible and consistent as heuristic function.

\section{Implementation of Problem Description}
The state of the problem is composed of the current position of the agent, the position of the goal, the labyrinth and the state of the doors. For the $A^{*}$ algorithm, the accumulated path cost and the heuristic function value is also included in the state. The position of the agent and the labyrinth (except doors) is the same for every node. The variables for each state are included in a class together with the position of the father of a node. This position permits the comparison between positions of grandfather/grandson, and so it is possible to avoid creating a set of invalid nodes (with the same state), saving some computation time.

In the informed search, the path cost was implemented as the number of steps, including presses, since the initial position of the agent. This was chosen taking into account that this was what should be optimized.

A successor function was also implemented, which returns the possible movements of the agent, given its position and the state of the doors. To increase the performance of the developed program, this function does not return a switch press as a possible movement when all the doors correspondent to the switch are open.

The goal state is verified as soon as the new position from a given state and movement is computed. If this is verified no more nodes are generated, and the solution is presented in the terminal and saved to a file.


\section{Results and Discussion}
Tab. \todo[inline]{ref} presents the results in terms of generated nodes for each method implemented for all the eight labyrinths in which the analysis was based. The processing time is not indicated here since the results obtained in that term were not significant for comparison. Nevertheless, note for the fact that both methods could solve the analysed labyrinths in less than one second.
\begin{table}[h]
\centering
\begin{tabular}{c|cc|cc}
  & \multicolumn{2}{c|}{Nodes Generated} & \multicolumn{2}{c}{Computation Time} \\
  \hline
 Map &      Breadth-First     &    A*      &      Breadth-First     &         A* \\
 1&           &          &           &          \\
 2&           &          &           &          \\
 3&           &          &           &          \\
 4&           &          &           &          \\
 5&           &          &           &          \\
 6&           &          &           &         
\end{tabular}
\end{table}
% An example of a floating figure using the graphicx package.
% Note that \label must occur AFTER (or within) \caption.
% For figures, \caption should occur after the \includegraphics.
% Note that IEEEtran v1.7 and later has special internal code that
% is designed to preserve the operation of \label within \caption
% even when the captionsoff option is in effect. However, because
% of issues like this, it may be the safest practice to put all your
% \label just after \caption rather than within \caption{}.
%
% Reminder: the "draftcls" or "draftclsnofoot", not "draft", class
% option should be used if it is desired that the figures are to be
% displayed while in draft mode.
%
%\begin{figure}[!t]
%\centering
%\includegraphics[width=2.5in]{myfigure}
% where an .eps filename suffix will be assumed under latex, 
% and a .pdf suffix will be assumed for pdflatex; or what has been declared
% via \DeclareGraphicsExtensions.
%\caption{Simulation Results.}
%\label{fig_sim}
%\end{figure}

% Note that IEEE typically puts floats only at the top, even when this
% results in a large percentage of a column being occupied by floats.
% However, the Computer Society has been known to put floats at the bottom.


% An example of a double column floating figure using two subfigures.
% (The subfig.sty package must be loaded for this to work.)
% The subfigure \label commands are set within each subfloat command,
% and the \label for the overall figure must come after \caption.
% \hfil is used as a separator to get equal spacing.
% Watch out that the combined width of all the subfigures on a 
% line do not exceed the text width or a line break will occur.
%
%\begin{figure*}[!t]
%\centering
%\subfloat[Case I]{\includegraphics[width=2.5in]{box}%
%\label{fig_first_case}}
%\hfil
%\subfloat[Case II]{\includegraphics[width=2.5in]{box}%
%\label{fig_second_case}}
%\caption{Simulation results.}
%\label{fig_sim}
%\end{figure*}
%
% Note that often IEEE papers with subfigures do not employ subfigure
% captions (using the optional argument to \subfloat[]), but instead will
% reference/describe all of them (a), (b), etc., within the main caption.


% An example of a floating table. Note that, for IEEE style tables, the 
% \caption command should come BEFORE the table. Table text will default to
% \footnotesize as IEEE normally uses this smaller font for tables.
% The \label must come after \caption as always.
%
%\begin{table}[!t]
%% increase table row spacing, adjust to taste
%\renewcommand{\arraystretch}{1.3}
% if using array.sty, it might be a good idea to tweak the value of
% \extrarowheight as needed to properly center the text within the cells
%\caption{An Example of a Table}
%\label{table_example}
%\centering
%% Some packages, such as MDW tools, offer better commands for making tables
%% than the plain LaTeX2e tabular which is used here.
%\begin{tabular}{|c||c|}
%\hline
%One & Two\\
%\hline
%Three & Four\\
%\hline
%\end{tabular}
%\end{table}


% Note that IEEE does not put floats in the very first column - or typically
% anywhere on the first page for that matter. Also, in-text middle ("here")
% positioning is not used. Most IEEE journals use top floats exclusively.
% However, Computer Society journals sometimes do use bottom floats - bear
% this in mind when choosing appropriate optional arguments for the
% figure/table environments.
% Note that, LaTeX2e, unlike IEEE journals, places footnotes above bottom
% floats. This can be corrected via the \fnbelowfloat command of the
% stfloats package.





% if have a single appendix:
%\appendix[Proof of the Zonklar Equations]
% or
%\appendix  % for no appendix heading
% do not use \section anymore after \appendix, only \section*
% is possibly needed

% use appendices with more than one appendix
% then use \section to start each appendix
% you must declare a \section before using any
% \subsection or using \label (\appendices by itself
% starts a section numbered zero.)
%


%\appendices
%\section{Proof of the First Zonklar Equation}
%Appendix one text goes here.

% you can choose not to have a title for an appendix
% if you want by leaving the argument blank
%\section{}
%Appendix two text goes here.


% use section* for acknowledgement
%\ifCLASSOPTIONcompsoc
  % The Computer Society usually uses the plural form
 % \section*{Acknowledgments}
%\else
  % regular IEEE prefers the singular form
 % \section*{Acknowledgment}
%\fi



% Can use something like this to put references on a page
% by themselves when using endfloat and the captionsoff option.
\ifCLASSOPTIONcaptionsoff
  \newpage
\fi



% trigger a \newpage just before the given reference
% number - used to balance the columns on the last page
% adjust value as needed - may need to be readjusted if
% the document is modified later
%\IEEEtriggeratref{8}
% The "triggered" command can be changed if desired:
%\IEEEtriggercmd{\enlargethispage{-5in}}

% references section

% can use a bibliography generated by BibTeX as a .bbl file
% BibTeX documentation can be easily obtained at:
% http://www.ctan.org/tex-archive/biblio/bibtex/contrib/doc/
% The IEEEtran BibTeX style support page is at:
% http://www.michaelshell.org/tex/ieeetran/bibtex/
%\bibliographystyle{IEEEtran}
% argument is your BibTeX string definitions and bibliography database(s)
%\bibliography{IEEEabrv,../bib/paper}
%
% <OR> manually copy in the resultant .bbl file
% set second argument of \begin to the number of references
% (used to reserve space for the reference number labels box)
\begin{thebibliography}{1}

\bibitem{book}
S.~Russell and P.~Norvig, \emph{Atificial Intelligence - A Modern Approach}, 3rd~ed.\hskip 1em plus
  0.5em minus 0.4em\relax Pearson, 2010.

\end{thebibliography}

% biography section
% 
% If you have an EPS/PDF photo (graphicx package needed) extra braces are
% needed around the contents of the optional argument to biography to prevent
% the LaTeX parser from getting confused when it sees the complicated
% \includegraphics command within an optional argument. (You could create
% your own custom macro containing the \includegraphics command to make things
% simpler here.)
%\begin{IEEEbiography}[{\includegraphics[width=1in,height=1.25in,clip,keepaspectratio]{mshell}}]{Michael Shell}
% or if you just want to reserve a space for a photo:


\end{document}


